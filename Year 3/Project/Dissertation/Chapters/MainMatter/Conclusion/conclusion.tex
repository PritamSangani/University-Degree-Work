% *****************************************************
%
% CONCLUSION - Briefly restate the work undertaken, 
% summarise any findings or recommendations,
% acknowledge limitations and make suggestions for 
% further work.
%
% *****************************************************
\chapter{Conclusion}
This chapter will review the project as a whole and summarise the findings from the evaluation and will include a personal evaluation of how well the author of this report completed this project before making some recommendations on further work that can improve this area of study.
\section{Project Review}
At the start of the project, the aims and objectives were set out to help define what the author wanted to achieve from the project. The aim of the project was to implement a long-term memory mechanism in the ELIZA chatterbot and evaluate the memory's performance. This formed the basis of the research question and hypotheses, which were: \\\\
\textbf{RQ: Is it possible to implement convincing long-term memory into an existing chatterbot, such as Weizenbaum's ELIZA?} \\\\
\textbf{H\textsubscript{0} - it is not possible to add long-term memory to a chatterbot, at this stage, which is convincing enough to be comparable to human memory mechanism.}\\\\
\textbf{H\textsubscript{1} - it is possible to add convincing long-term memory at a significant statistical level, if the number of participants in the research is high enough.}\\\\
Once the project aims and objectives had been decided, a literature review was completed to synthesise current research and decide on the most appropriate techniques and tooling to implement the product with that would help answer the proposed research question. The most important areas of research was in the areas of ELIZA, the psychology of conversation, including long-term memory and Named Entity Recognition, a popular Natural Language Processing technique. The research in the psychology of conversation helped provide guidance in the most prominent topics of conversation that people talk about and the research in the area of Named Entity Recognition helped in learning about how \gls{ner} works and the types of entities the technique can extract from textual data.\\\\
The design phase of the project involved planning and visualising how the system would look like. The first task was to list all the requirements of the project and categorise them into functional and non-functional requirements, which helped in defining the questions for the usability and functionality evaluation. Diagrams were also drawn up to visualise the schema for the NoSQL database and a system overview diagram to visualise how the different components of the system communicated with each other. The design phase helped to save a lot of time during the implementation phase as all of the requirements had been identified and the best design of the system had been researched, as well as the technologies and libraries that were most suitable for this project.\\\\
During the implementation phase, the design plans were followed closely, however, the methodology changed from Agile to Waterfall because it turned out to be difficult to implement and debug some features, which meant that enough features were not being implemented in the agile time frame of two weeks. Another reason why the Waterfall method was used is because it was quite hard to test individual features as a lot of features depended on other features. Therefore, it made sense to test the whole system as a whole at the end of the implementation phase, with some minor testing done along the way. The main difficulties in the implementation was learning how to use the libraries that were used, such as spaCy and socket.io, and also learning how to program efficiently in Python. It was also challenging to work out how to best approach and implement the interruption in conversation flow to access the memory, which contained the entities that had been extracted from the user's previous chat messages. Another key point to mention is that originally the mode of communication between the web interface and the chatterbot was to be via an API that was going to be built using the Express.js package in the Node.js runtime environment. However, this was taking too long to implement and a easier way, but less efficient and secure, was found in the form of web sockets - the design diagrams shown in Chapter \ref{chap:design} reflect this change.\\\\
To answer the research question and evaluate the hypotheses, a usability and functionality evaluation was held with 5 participants. The participants were give one of three scenarios, which tested each of the three entity types that were accounted for in the implementation. The entity types accounted for were names of products, names of places and names of people. The participants had two sessions with the chatterbot to test the accuracy and quality of the long-term memory mechanism. The overall feedback for the chatterbot was that the chatterbot was engaging, but the long-term memory mechanism didn't pick up every key piece of information that was talked about and when the chatterbot referred to topics talked about in the past, sometimes the chatterbot moved on too quickly to other topics and didn't probe fully about key topics. These limitations are both down to the accuracy of models that implement \gls{ner} and the naive approach that was taken to implement the conversation flow, due to the complex nature of conversation.\\\\
From the evaluation completed, it can be evaluated that hypothesis H\textsubscript{0} is the most true and more work is needed to move towards a more human-like memory mechanism.

\section{Personal Evaluation}
This section is dedicated to allow the author to personally evaluate his performance during the project and allow him to identify his shortcomings and successes to better his performance in subsequent projects.\\\\
% i will evaluate my performance, what i did bad =, what i did good.
I felt my performance on this project had a lot of positives, but I acknowledge that I had some shortcomings along the way, which meant that the quality of the product was not as high as it could have or should have been. I feel that I did not plan the project with enough detailed plans, which meant that I had to change my approach a couple of times during the implementation period. I also should have synthesised my literature study, when I did it at the start of the project as I had to look back at the same papers during the implementation. Doing this would have saved me a lot of time, both during the implementation and during the writing of this report. Another shortcoming is that I didn't follow the implementation plan and so quickly fell behind on my work for this project. \\\\
However, there are a lot of positives to take out of this experience. Firstly, I can say that I am proud of learning how to program in Python in such a short amount of time and also I have improved my project management skills, which I can transfer and use throughout my career. I have also learnt a lot about the interesting and fast growing area of \gls{nlp} and can see the importance of the research that has already been done and will be done in the future. \\\\
Some skills that I have developed are project management, programming and problem solving through managing this project, building the long-term memory mechanism in Python and I developed my problem solving through learning a new language and working with a new \gls{ai} technique that I hadn't used before. Some skills that I have identified that need work are time management and project management. I plan to work on these 2 skills by working on a few projects after my exams are over and taking the time to plan out each project properly and setting out timelines as to when I should finish certain tasks. I also need to work on working out how to keep myself motivated when I run into technical or implementation difficulties, as I lost motivation when I ran into problems during this project.
\section{Recommendations for further work}
% give some recomendations for further work and explain why and what research is needed.
To take this project further, more work should be done in researching more suitable data structures to hold and connect different pieces of data together. This is because at the moment, the product that has been implemented doesn't implement a suitable data structure to hold multiple pieces of information about a single topic and so this hindered the dialogue manager to bring in suitable information about a topic that was previously talked about. Another area of research that must be looked into further is in dialogue management and how to probe for more information about a topic and then link the newly found information with the topic as that was one of the shortcomings of this project. Finally, a topic that wasn't looked at during the project but could help identify the sentiment that users are feeling is semantic analysis, which is another technique that falls under the area of \gls{nlp}. This could help in showing and understanding emotion, which can help make the chatterbot more human-like.
\section{Summary}
This chapter reviewed the project as a whole, including what went well and where improvements could be made. It first reviewed the aims and objectives defined at the start of the project and how the project helped achieve them. The author then personally evaluated his performance during the project and where there were shortcomings as well as successes. Finally, some recommendations were made for further work that could be carried out to take this project further.