% Notes to take into exam

% ********************************************
%
% PREAMBLE - DEFINE ALL STYLES AND CLASSES 
% AND PACKAGES REQUIRED THROUGHOUT DOCUMENT
%
% ********************************************
\documentclass[12pt,a4paper,openany,oneside]{book}
\linespread{1.25}
\usepackage[hyphens,spaces,obeyspaces]{url}
% PDF METADATA 
\usepackage[pdftex, 
pdfauthor={Pritam Sangani},
pdftitle={ELIZA Plus - Long Term Memory}]{hyperref}
\usepackage{setspace}
\usepackage{lmodern}

\pagestyle{plain}
% MARGIN STYLES
\usepackage{geometry}
\geometry{a4paper,
total={170mm,257mm},
top=25mm,
bottom=25mm,
left=25mm, 
right=25mm}

% PACKAGES NEEDED
\usepackage{graphicx} % for including images in document
\graphicspath{ {./Images/} }

%bibliography managing package
\usepackage[citestyle=authoryear,style=authoryear, sorting=nyt]{biblatex}

\DeclareBibliographyCategory{cited}
\AtEveryCitekey{\addtocategory{cited}{\thefield{entrykey}}}
\addbibresource{Chapters/BackMatter/References/references.bib}
\nocite{*}
\usepackage[nottoc, notlot, notlof]{tocbibind}


%Other useful packages
\usepackage{lscape} %to make sections of the document landscape
\usepackage{lipsum}

% chapter heading style for frontmatter
\newcommand{\cchapter}[1]{\chapter[#1]{\centering \Large #1}}

% command to stop from adding to contents
\newcommand{\nocontentsline}[3]{}
\newcommand{\tocless}[2]{\bgroup\let\addcontentsline=\nocontentsline#1{#2}\egroup}

% package to manage acronyms
\usepackage{longtable}
\usepackage{datatool}
\usepackage[acronym]{glossaries}
\setglossarystyle{long}
\renewcommand{\glsnamefont}[1]{\textbf{#1}}

\usepackage{pdfpages}
\usepackage{float}
\usepackage{fancyvrb}


\usepackage{fmtcount}

\newcommand\countnumwords{%
	\immediate\write18{texcount -1 -sum -inc \jobname.tex -out=\jobname.txt}
	\newread\tmp
	\openin\tmp=\jobname.txt
	\read\tmp to \words
	\gdef\wordcount{\numexpr \words\relax}
	\closein\tmp	
}


\usepackage{appendix}

\usepackage{color}
\definecolor{deepblue}{rgb}{0,0,0.5}
\definecolor{deepred}{rgb}{0.6,0,0}
\definecolor{deepgreen}{rgb}{0,0.5,0}

\DeclareFixedFont{\ttb}{T1}{txtt}{bx}{n}{12} % for bold
\DeclareFixedFont{\ttm}{T1}{txtt}{m}{n}{12}  % for normal

\usepackage{listings}

\newcommand\pythonstyle{\lstset{
		language=Python,
		basicstyle=\ttm,
		otherkeywords={self},             % Add keywords here
		keywordstyle=\ttb\color{deepblue},
		emph={MyClass,__init__},          % Custom highlighting
		emphstyle=\ttb\color{deepred},    % Custom highlighting style
		stringstyle=\color{deepgreen},
		frame=tb,                         % Any extra options here
		showstringspaces=false            % 
}}

\lstnewenvironment{python}[1][]
{
	\pythonstyle
	\lstset{#1}
}
{}

\usepackage{array}

\newcommand{\findabbreviations}{
	\immediate\write18{makeglossaries.exe \jobname}
}
\title{\vspace{-3.0cm}\textbf{Programming Principles Exam Notes}}
\date{}
\begin{document}
	\maketitle
	\section*{Section A}
	\subsection*{Lexical Analysis}
	Write regular expressions and include fragments if including them in the answer.\\\\
	\textbf{Things to remember:}
	\begin{itemize}
		\item ? = zero or one
		\item * = zero or many
		\item + = one or many
		\item Capitalize name of token and remember to add semicolon to end of rule.
	\end{itemize}
	\textbf{Examples:}\\
	ID: [a-zA-Z\_] [a-zA-Z0-9\_]*;\\
	NUMBER: [0-9]+ | `0x' [0-9a-fA-F]+;\\\\
	\subsection*{Parser}
	Write corresponding ANTLR rules given part of the Decaf grammar.\\
	\textbf{Things to remember:}
	\begin{itemize}
		\item Use rules in answer if given them.
		\item include any rules you have created or assumed exist in lexer. E.g. `COLON: `:'; '
		\item Carefully read grammar to see if it uses any of the meta-notation which will be given in a table.
	\end{itemize}
	\textbf{Example:}\\
	<method\_decl>  -> \{<type> | void\} <id> ([\{<type> <id>\}$^{+}$, ]) <block>\\
	<method\_type>  -> \{<type> | void\}\\
	<parameter>  -> <type> <id>\\\\
	\noindent
	\textbf{Answer:}\\ 
	LBRACKET : `(';\\
	RBRACKET : `)';\\
	COMMA : `,'; \\
	method\_decl: method\_type  id LBRACKET (parameter (COMMA parameter)*)? RBRACKET block;\\
\end{document}