% Notes to take into exam

% ********************************************
%
% PREAMBLE - DEFINE ALL STYLES AND CLASSES 
% AND PACKAGES REQUIRED THROUGHOUT DOCUMENT
%
% ********************************************
\documentclass[11pt,a4paper]{article}
\usepackage[hyphens,spaces,obeyspaces]{url}
% PDF METADATA 
\usepackage[pdftex, 
pdfauthor={Pritam Sangani},
pdftitle={Feasibility Study}]{hyperref}

% WATERMARK STYLES
\usepackage{draftwatermark}
\SetWatermarkScale{2}

%%% Macro to disable watermark
\makeatletter
\def\watermarkoff{%
	\@sc@wm@stampfalse
}
\makeatother

%%% Macro to enable watermark
\makeatletter
\def\watermarkon{%
	\@sc@wm@stamptrue
}
\makeatother


% MARGIN STYLES
\usepackage{geometry}
\geometry{a4paper,
total={170mm,257mm},
top=25mm,
bottom=25mm}

% PACKAGES NEEDED
\usepackage{graphicx} % for including images in document
\graphicspath{ {./Images/} }


%header
\usepackage{fancyhdr}
\pagestyle{fancy}
\lhead{Feasibility Study: Eliza Plus - Long-Term Memory}
\rhead{October 2018}

%bib
\usepackage{natbib}
\bibliographystyle{agsm}

%gannt chart package
\usepackage{pgfgantt}

\newcounter{myWeekNum}
\stepcounter{myWeekNum}
%
\newcommand{\myWeek}{\themyWeekNum
	\stepcounter{myWeekNum}
	\ifnum\themyWeekNum=53
	\setcounter{myWeekNum}{1}
	\else\fi
}

%Other useful packages
\usepackage{lscape} %to make sections of the document landscape


\title{\vspace{-3.0cm}\textbf{Programming Principles Exam Notes}}
\date{}
\begin{document}
	\maketitle
	\section*{Section A}
	\subsection*{Lexical Analysis}
	Write regular expressions and include fragments if including them in the answer.\\\\
	\textbf{Things to remember:}
	\begin{itemize}
		\item ? = zero or one
		\item * = zero or many
		\item + = one or many
		\item Capitalize name of token and remember to add semicolon to end of rule.
	\end{itemize}
	\textbf{Examples:}\\
	ID: [a-zA-Z\_] [a-zA-Z0-9\_]*;\\
	NUMBER: [0-9]+ | `0x' [0-9a-fA-F]+;\\\\
	\subsection*{Parser}
	Write corresponding ANTLR rules given part of the Decaf grammar.\\
	\textbf{Things to remember:}
	\begin{itemize}
		\item Use rules in answer if given them.
		\item include any rules you have created or assumed exist in lexer. E.g. `COLON: `:'; '
		\item Carefully read grammar to see if it uses any of the meta-notation which will be given in a table.
	\end{itemize}
	\textbf{Example:}\\
	<method\_decl>  -> \{<type> | void\} <id> ([\{<type> <id>\}$^{+}$, ]) <block>\\
	<method\_type>  -> \{<type> | void\}\\
	<parameter>  -> <type> <id>\\\\
	\noindent
	\textbf{Answer:}\\ 
	LBRACKET : `(';\\
	RBRACKET : `)';\\
	COMMA : `,'; \\
	method\_decl: method\_type  id LBRACKET (parameter (COMMA parameter)*)? RBRACKET block;\\
\end{document}