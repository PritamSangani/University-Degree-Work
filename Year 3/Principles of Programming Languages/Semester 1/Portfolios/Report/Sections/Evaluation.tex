\section*{Evaluation and Reflection}
\subsection*{Pharo}
I found Pharo a bit tricky at first as I had not experienced a language with syntax like Pharo's. However, the cheat sheet on Pharo Docs made it easier to understand and and get to grips with the syntax. I did not like the IDE as it was quite unstable and it crashed a couple of times. I also did not like that it ran on a Virtual Machine as it meant that my laptop slowed down and that if the IDE crashed or I did not save my work to my host system, I could not recover my work. I also did not like that there wasn't a comprehensive documentation website as the Pharo Docs only listed the books and the cheat sheet. However, the books were useful as they contained a lot of examples.
\subsection*{Ruby}
I enjoyed programming in Ruby as the syntax made my code rally easy to write and elegant looking. I also liked that there was good documentation online which explained fully, with examples, how to use the various core functions and libraries. I also like the IRB as it allowed me to evaluate and test that my functions work very easily.
\subsection*{JavaScript}
I enjoyed programming in JavaScript as I have programmed in it quite a bit previously and it allowed me to use the portfolio as a refresher, as well as, giving me the opportunity to practice writing JavaScript using the ES2016 standard, which I hadn't used previously. I like the JavaScript documentation, with my favourite and goto being the documentation on MDN.
\subsection*{Clojure}
I found it quite difficult to code in Clojure as this was my first experience of programming in a functional language. It took some time to get used to thinking in a much different way than I do when programming in imperative language and also getting used to putting brackets on the outside of functions. I also found it weird that the operation argument order was like it is in Clojure with the operand being first followed by its arguments. However, once I got used to the way of thinking I enjoyed structuring my code in a way that it was easy to work myself up the chain of functions evaluating the expressions until I got to the function prototype which evaluates the final result. I found that this is a much better way of writing reliable and easy to understand code.
\subsection*{Haskell}
As I had already written code in Clojure, this was less of a steep learning curve and I focussed more on researching and using parts of Haskell that was not taught in the lectures as I enjoy writing the best and most efficient code that I can. I enjoyed refactoring my code to use function applications and compositions as I could see how much cleaner and more maintainable my code became.
\subsection*{Overall Reflection}
Overall, I found that I have many strengths that I discovered during the course of this semester. One of these being my ability to pick up new languages pretty quickly and being able to interpret documentation and find relevant help online on forums such as Stack Overflow when the documentation does not answer my questions. I also discovered some weaknesses as well that I need to work on. One of these was that I often got fixated on finding unique ways of implementing a particular function or operation. I often spent a lot of time on finding a solution when I could have been working on something else. However, I found that if I spent some time away from the problem, this cleared my head and I often found that I quickly found a solution when I came back to solving the problem.